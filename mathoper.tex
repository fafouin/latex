%% define additional mathematical operators and LaTeX commands %%

% new math operators names (starred to use subscripts or superscripts)
\DeclareMathOperator{\sinc}{sinc} % cardinal sine
\DeclareMathOperator{\rect}{rect} % rectangular window
\DeclareMathOperator*{\E}{\mathsf{E}} % expectation
\DeclareMathOperator{\var}{Var} % variance
\DeclareMathOperator{\cov}{Cov} % covariance
\DeclareMathOperator{\tr}{Tr} % trace

% new mathematical symbols
\newcommand*{\I}{\mathrm{i}} % upright imaginary i symbol
\newcommand*{\J}{\mathrm{j}} % upright imaginary j symbol
\newcommand*{\e}{\mathrm{e}} % upright Euler (e) symbol
\newcommand*{\dif}{\mathrm{d}} % upright derivative symbol

% mathematical delimiters
\newcommand*{\abs}[1]{\lvert#1\rvert} % absolute value
\newcommand*{\norm}[1]{\lVert#1\rVert} % Euclidian norm
\newcommand*{\avg}[1]{\left<#1\right>} % average value
\newcommand*{\bp}[1]{\left(#1\right)} % pair of parentheses
\newcommand*{\bsq}[1]{\left[#1\right]} % pair of square brackets
\newcommand*{\bbr}[1]{\left\{#1\right\}} % pair of curly braces
\newcommand*{\evalf}[2]{\left.#1\right|_{#2}} % function evaluation

% miscellaneous
\newcommand*{\V}[1]{\boldsymbol{\mathbf{#1}}} % vector
\newcommand*{\uV}[1]{\boldsymbol{\mathbf{\hat{#1}}}} % unit vector
\newcommand*{\D}[2]{\frac{\dif #1}{\dif #2}} % derivative
\newcommand*{\pD}[2]{\frac{\partial #1}{\partial #2}} % partial derivative
\newcommand*{\Dn}[3]{\frac{\dif^#3 #1}{\dif #2^#3}} % order-n derivative
\newcommand*{\pDn}[3]{\frac{\partial^#3 #1}{\partial #2^#3}}
\newcommand*{\grad}[1]{\V{\nabla} #1} % gradient of argument
\newcommand*{\diver}[1]{\V{\nabla} \cdot #1} % divergence of argument
\newcommand*{\curl}[1]{\V{\nabla} \times #1} % curl of argument

